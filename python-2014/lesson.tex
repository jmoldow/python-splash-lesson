% $Header: /cvsroot/latex-beamer/latex-beamer/solutions/generic-talks/generic-ornate-15min-45min.en.tex,v 1.5 2007/01/28 20:48:23 tantau Exp $

% This file is a solution template for:

% - Giving a talk on some subject.
% - The talk is between 15min and 45min long.
% - Style is ornate.



% Copyright 2004 by Till Tantau <tantau@users.sourceforge.net>.
%
% In principle, this file can be redistributed and/or modified under
% the terms of the GNU Public License, version 2.
%
% However, this file is supposed to be a template to be modified
% for your own needs. For this reason, if you use this file as a
% template and not specifically distribute it as part of a another
% package/program, I grant the extra permission to freely copy and
% modify this file as you see fit and even to delete this copyright
% notice. 


\mode<presentation>
{
  \usetheme{Warsaw}
  % or ...

  \setbeamercovered{transparent}
  % or whatever (possibly just delete it)
}

\mode<article>
{
  \usepackage{fullpage,hyperref}
  \hypersetup{
  colorlinks,%
  citecolor=black,%
  filecolor=black,%
  linkcolor=black,%
  urlcolor=black,
  pdftitle={Introduction to Python - MIT ESP},
  pdfauthor={Jordan Moldow}
  }
}

\usepackage[english]{babel}
% or whatever

\usepackage[latin1]{inputenc}
% or whatever

\usepackage{times}
\usepackage[T1]{fontenc}
\usepackage{amsmath,amssymb,amsfonts}
% Or whatever. Note that the encoding and the font should match. If T1
% does not look nice, try deleting the line with the fontenc.


\title % (optional, use only with long paper titles)
{Introduction to Python}

\subtitle
{} % (optional)

\author[Jordan Moldow] % (optional, use only with lots of authors)
{Jordan Moldow
}
% - Use the \inst{?} command only if the authors have different
%   affiliation.

\institute[MIT ESP] % (optional, but mostly needed)
{
  MIT Educational Studies Program}
% - Use the \inst command only if there are several affiliations.
% - Keep it simple, no one is interested in your street address.

\date[] % (optional)
{}

%\subject{Talks}
% This is only inserted into the PDF information catalog. Can be left
% out. 



% If you have a file called "university-logo-filename.xxx", where xxx
% is a graphic format that can be processed by latex or pdflatex,
% resp., then you can add a logo as follows:

%\pgfdeclareimage[height=0.5cm]{logo}{../logo.png}
%\logo{\pgfuseimage{logo}}



% Delete this, if you do not want the table of contents to pop up at
% the beginning of each subsection:
\AtBeginSubsection[]
{
  \begin{frame}<beamer>{Outline}
    \tableofcontents[currentsection,currentsubsection]
  \end{frame}
}


% If you wish to uncover everything in a step-wise fashion, uncomment
% the following command: 

%\beamerdefaultoverlayspecification{<+->}

\begin{document}
\maketitle
\begin{frame}
  \titlepage
\end{frame}

%\begin{frame}{\only<presentation>{Outline}}
%  
%  % You might wish to add the option [pausesections]
%\end{frame}
\tableofcontents
%\pagebreak
\section{Introduction to Programming}

\begin{frame}{Programming}
\begin{itemize}
\item A program is ``a sequence of coded instructions for a computer''
\item Programming is the coding of these instructions by humans
\item ``The purpose of programming is to create a program that exhibits a certain desired behavior.''
\item Programming is ``writing the source code of computer programs''
\end{itemize}
\end{frame}

\begin{frame}{General Programming Steps}
\begin{enumerate}
\item Pick a programming language
\item Write ``source code'' inside a text file
\begin{itemize}
\item Source code is understandable by humans [who know the language]
\item Each language has different code syntax
\end{itemize}
\item (For compiled languages) A ``compiler'' translates source into binary / machine code that is understandable by computers
\item Computer executes code
\end{enumerate}
\end{frame}

\begin{frame}{Writing Python Programs}
\begin{itemize}
\item Code can be written and saved using special programming environments -- file type is .py
\item Code can also be written in a normal text editor
\begin{itemize}
\item *nix: vim, emacs, gedit, NOT OpenOffice or LibreOffice
\item Windows: Notepad, Notepad++, NOT Word
\end{itemize}
\item We will use Python IDLE, an officially supported integrated development environment
\item Python interpreter executes code directly from your source code -- Python is an \textbf{interpretted language}
\end{itemize}
\end{frame}

\begin{frame}{Python Interpreter}
\begin{itemize}
\item The first thing you see after opening Python IDLE is a command prompt
\item This is a shell for the Python interpreter
\item Go ahead and type stuff into it
\item In its most basic form, the interpreter acts like a calculator, supporting all basic mathematical operations and orders of operations
\item Of course, the shell is infinitely more powerful than this, and we will slowly build up our knowledge of what Python can do
\end{itemize}
\end{frame}

\begin{frame}{Writing and Saving Programs}
\begin{itemize}
\item No code you write into the interpreter is permanent -- it will be lost when you close the interpreter
\item You can save code into a file so that you can run it whenever you want
\item In Python IDLE, File -> New Window opens a Python file, which you can write code into, save, and run
\end{itemize}
\end{frame}

\begin{frame}{Hello World! Your First Program!}
\begin{itemize}
\item A programming tradition -- your first program simply outputs the text \textbf{Hello World!}
\item ``Output'', in this and most cases, means to write text on the screen
\end{itemize}
\end{frame}

%\begin{frame}{hello.cpp}
%\verbatim{
%# Program: hello.py
%print "Hello World!"
%}
%\end{frame}

\end{document}
